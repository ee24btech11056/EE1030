%iffalse
\let\negmedspace\undefined
\let\negthickspace\undefined
\documentclass[journal,12pt,twocolumn]{IEEEtran}
\usepackage{cite}
\usepackage{amsmath,amssymb,amsfonts,amsthm}
\usepackage{algorithmic}
\usepackage{graphicx}
\usepackage{textcomp}
\usepackage{xcolor}
\usepackage{txfonts}
\usepackage{listings}
\usepackage{enumitem}
\usepackage{mathtools}
\usepackage{gensymb}
\usepackage{comment}
\usepackage[breaklinks=true]{hyperref}
\usepackage{tkz-euclide} 
\usepackage{listings}  
\usepackage{gvv}
%\def\inputGnumericTable{}

\usepackage[latin1]{inputenc}                                
\usepackage{color}                                            
\usepackage{array}                                            
\usepackage{longtable}                                       
\usepackage{calc}                                             
\usepackage{multirow}
\usepackage{multicol}
\usepackage{hhline}                                           
\usepackage{ifthen}                                           
\usepackage{lscape}
\usepackage{tabularx}
\usepackage{array}
\usepackage{float}


\newtheorem{theorem}{Theorem}[section]
\newtheorem{problem}{Problem}
\newtheorem{proposition}{Proposition}[section]
\newtheorem{lemma}{Lemma}[section]
\newtheorem{corollary}[theorem]{Corollary}
\newtheorem{example}{Example}[section]
\newtheorem{definition}[problem]{Definition}
\newcommand{\BEQA}{\begin{eqnarray}}
\newcommand{\EEQA}{\end{eqnarray}}
\newcommand{\define}{\stackrel{\triangle}{=}}
\theoremstyle{remark}
\newtheorem{rem}{Remark}

% Marks the beginning of the document
\begin{document}
\bibliographystyle{IEEEtran}
\vspace{3cm}

\title{Assignment(matrix theory)}
\author{ee24btech11056 - S.Kavya Anvitha}
\maketitle
\textbf{A Fill in the blanks}\\
1.  The larger of $\cos $\brak{\ln \theta}$ $ and  $\ln $\brak{\cos \theta}$ $ if
$e^{\frac{-\pi}{2}}< \theta< \frac{\pi}{2}$\\
\indent is ..........\hfill$\brak{1983 - 1 Mark}$\\
 2.  The function $y=2x^{2}-\ln \abs x$ is monotonically \indent increasing
 for values of $x$\brak{\neq0}$ $ satisfying the \indent inequalities ...... and
 monotonically decreasing for \indent values of x satisfying the inequalities 
 ............\\ \indent\hfill$\brak{1983 - 2 Marks}$\\
3.  The set of all x for which $\ln $\brak{1+x}$ \leq x$ is equal \indent
to .........\hfill$\brak{1987 - 2 Marks}$\\
4.  Let P be a variable point on the ellipse
$\frac{x^2}{a^2}+\frac{y^2}{b^2} = 1$\indent with foci 
$F_1$ and $F_2$. If A is the area of the \indent triangle 
P$F_1$$F_2$ then the maximum value of A is \hspace{5cm}\indent............. 
\hfill$\brak{1994 - 2 Marks}$\\
5.  Let C be the curve $y^3$ - 3xy + 2 = 0. If H is the \indent set of
points on the curve C where the tangent \indent is horizontal and V is
the set of the point on the \indent curve C where the tangent is vertical
then H = \indent ......... and V = .........\hfill$\brak{1994 - 2 Marks}$\\
\indent\hspace{0.3cm}\textbf{B True / False}\\
1. If x-r is a factor of the polynomial f(x) = \indent $a_n$$x^4$+....+$a_0$,
repeated m times $ $\brak{1< m\leq n}$ $, then \indent r is a root of f'$\brak{x}$=0 
repeated m times.\\
\indent \hfill$\brak{1983 - 1 Mark}$\\
2.  For $0 < a < x$, the minimum value of the function \indent 
$log_a x + log_x a$ is 2. \hfill$\brak{1984 - 1 Mark}$\\
\indent\hspace{0.3cm}\textbf{C MCQs with One Correct Answer}\\
1.  If a+b+c = 0, then the quadratic equation 3a$x^2$ \indent + 2bx + c = 0 
has\hfill$\brak{1983 - 1 Mark}$
\begin{enumerate}[label=\alph*.]
	\item at least one root in [0,1]
	\item one root in [2,3] and other in [-2,-1]
        \item imaginary roots
	\item none of these\\
\end{enumerate}
2.  AB is a diameter of a circle and C is any point \indent on the
circumference of the circle. Then\\ \indent\hfill
$\brak{1983 - 1 Mark}$
\begin{enumerate}[label=\alph*.]
	\item the area of$\Delta$ ABC is maximum when it is isosceles
	\item the area of $\Delta$ ABC is minimum when it is isosceles
	\item the perimeter of $\Delta$ ABC is minimumwhen it is isosceles
	\item none of these\\
\end{enumerate}
3.  The normal to the curve x = a($\cos \theta$ + $\theta\sin \theta$),
\indent y = a($\sin \theta$ - $\theta\cos \theta$)
at any point '$\theta$' 
is such that\\\indent\hfill$\brak{1983 - 1 Mark}$
\begin{enumerate}[label=\alph*.]
	\item it makes  constant angle with the x - axis
	\item it passes through the origin
	\item it is at a constant distance from the origin
	\item none of these\\
\end{enumerate}
4.  If $y=a\ln x + bx^2 +x$ has its extremum values at \indent x = -1
and x = 2, then\hfill$\brak{1983 - 1 Mark}$
\begin{enumerate}[label=\alph*.]
\begin{multicols}{2}
	\item a = 2, b = -1
	\item a = 2, b = $\displaystyle\frac{-1}{2}$
	\item a = -2, b = $\displaystyle\frac{1}{2}$
	\item none of these
\end{multicols}
\end{enumerate}
5. Which one of the following curves cut the \indent  parabola
$y^2 = 4ax$ at right angles?\hfill$\brak{1994}$
\begin{enumerate}[label=\alph*.]
\begin{multicols}{2}
	\item $x^2 + y^2 = a^2$
        \item $e^{\frac{-x}{2a}}$
	\item y = ax
	\item $x^2 = 4ay$
\end{multicols}
\end{enumerate}
6.  The function defined by f$\brak{x}$ = $\brak{x+2}$ $e^{-x}$ 	
is\hfill$\brak{1994}$
\begin{enumerate}[label=\alph*.]
	\item decreasing for all x
	\item decreasing in $ $\brak{-\infty, -1}$ $ and increasing
		in $ $\brak{-1, \infty}$ $
        \item increasing for all x
	\item decreasing in $ $\brak{-1, \infty}$ $ and increasing
		in $ $\brak{-\infty, -1}$ $\\
\end{enumerate}
7.  The function f$\brak{x}$ = 
$\displaystyle\frac{\ln (\pi + x)}{\ln (e + x)}$ 
is\hfill$\brak{1995S}$\\
\begin{enumerate}[label=\alph*.]
	\item increasing on $ $\brak{0, \infty}$ $
	\item decreasing on $ $\brak{0, \infty}$ $
	\item increasing on $ $\brak{0, \displaystyle\frac{\pi}{e}}$ $,
		decreasing on $ $\brak{\displaystyle\frac{\pi}{e}, \infty}$ $
	\item decreasing on $ $\brak{0, \displaystyle\frac{\pi}{e}}$ $,
		increasing on $ $\brak{\displaystyle\frac{\pi}{e}, \infty}$ $\\
\end{enumerate}
8.  On the interval [0, 1] the function $x^{25}$\brak{1-x}$^{25}$ \indent takes 
its maximum value at the point\hfill$\brak{1995S}$
\begin{enumerate}[label=\alph*.]
\begin{multicols}{4}
	\item 0 
	\item $\displaystyle\frac{1}{4}$ 
	\item $\displaystyle\frac{1}{2}$ 
        \item $\displaystyle\frac{1}{3}$
\end{multicols}
\end{enumerate}
\newpage
\bigskip
\renewcommand{\thefigure}
{\theenumi}
\renewcommand{\thetable}
{\theenumi}

\end{document}


