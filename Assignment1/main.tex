%iffalse
\let\negmedspace\undefined
\let\negthickspace\undefined
\documentclass[journal,12pt,twocolumn]{IEEEtran}
\usepackage{cite}
\usepackage{amsmath,amssymb,amsfonts,amsthm}
\usepackage{algorithmic}
\usepackage{graphicx}
\usepackage{textcomp}
\usepackage{xcolor}
\usepackage{txfonts}
\usepackage{listings}
\usepackage{enumitem}
\usepackage{mathtools}
\usepackage{gensymb}
\usepackage{comment}
\usepackage[breaklinks=true]{hyperref}
\usepackage{tkz-euclide} 
\usepackage{listings}  
\usepackage{gvv}
%\def\inputGnumericTable{}

\usepackage[latin1]{inputenc}                                
\usepackage{color}                                            
\usepackage{array}                                            
\usepackage{longtable}                                       
\usepackage{calc}                                             
\usepackage{multirow}
\usepackage{multicol}
\usepackage{hhline}                                           
\usepackage{ifthen}                                           
\usepackage{lscape}
\usepackage{tabularx}
\usepackage{array}
\usepackage{float}


\newtheorem{theorem}{Theorem}[section]
\newtheorem{problem}{Problem}
\newtheorem{proposition}{Proposition}[section]
\newtheorem{lemma}{Lemma}[section]
\newtheorem{corollary}[theorem]{Corollary}
\newtheorem{example}{Example}[section]
\newtheorem{definition}[problem]{Definition}
\newcommand{\BEQA}{\begin{eqnarray}}
\newcommand{\EEQA}{\end{eqnarray}}
\newcommand{\define}{\stackrel{\triangle}{=}}
\theoremstyle{remark}
\newtheorem{rem}{Remark}

% Marks the beginning of the document
\begin{document}
\bibliographystyle{IEEEtran}
\vspace{3cm}

\title{Assignment(matrix theory)}
\author{ee24btech11056 - S.Kavya Anvitha}
\maketitle
\textbf{A Fill in the blanks}\\
\begin{enumerate}
	\item The larger of $\cos \brak{\ln \theta}$ and 
$\ln \brak{\cos \theta}$ if 
$e^{\frac{-\pi}{2}}< \theta< \frac{\pi}{2}$
is ..........
		\hfill$\brak{1983 - 1 Mark}$\\

	\item The function \begin{align*}y=2x^{2}-\ln \abs x\end{align*}
is monotonically increasing for values of $x\brak{\neq0}$ satisfying
the inequalities ...... and monotonically decreasing for values of x 
satisfying the inequalities  ............

		\hfill$\brak{1983 - 2 Marks}$\\

	\item The set of all x for which \begin{align*}\ln \brak{1+x} \leq x\end{align*} is equal 
to .........\hfill$\brak{1987 - 2 Marks}$\\

         \item Let P be a variable point on the ellipse
\begin{align*}\frac{x^2}{a^2}+\frac{y^2}{b^2} = 1\end{align*}
with foci $F_1$ and $F_2$. If A is the area of the triangle P$F_1$$F_2$ 
then the maximum value of A is ............. \hfill$\brak{1994 - 2 Marks}$\\

         \item Let C be the curve \begin{align*}y^3 - 3xy + 2 = 0.\end{align*} If H is the set of
points on the curve C where the tangent is horizontal and V is
the set of the point on the curve C where the tangent is vertical
then H = ......... and V = .........\hfill$\brak{1994 - 2 Marks}$\\
\end{enumerate}

\textbf{B True / False}\\

\begin{enumerate}

	\item If $x-r$ is a factor of the polynomial
		\begin{align*}f\brak{x} = a_nx^4+....+a_0\end{align*},
repeated m times $\brak{1< m\leq n}$, then r is a root of 
$f'\brak{x}=0$ repeated m times.

\hfill$\brak{1983 - 1 Mark}$\\

         \item For $0 < a < x$, the minimum value of the function 
$log_a x + log_x a$ is $2$. \hfill$\brak{1984 - 1 Mark}$\\

\end{enumerate}

\textbf{C MCQs with One Correct Answer}\\

\begin{enumerate}

        \item If $a+b+c = 0$, then the quadratic equation 
		\begin{align*}3ax^2 + 2bx + c = 0\end{align*} has\hfill$\brak{1983 - 1 Mark}$

\begin{enumerate}
	\item at least one root in [0,1]
	\item one root in [2,3] and other in [-2,-1]
        \item imaginary roots
	\item none of these\\
\end{enumerate}

         \item AB is a diameter of a circle and C is any point on the
circumference of the circle. Then 

		\hfill$\brak{1983 - 1 Mark}$

\begin{enumerate}
	\item the area of$\Delta$ ABC is maximum when it is isosceles
	\item the area of $\Delta$ ABC is minimum when it is isosceles
	\item the perimeter of $\Delta$ ABC is minimumwhen it is isosceles
	\item none of these\\
\end{enumerate}

         \item The normal to the curve 
\begin{align*}x = a\brak{\cos \theta + \theta\sin \theta}\end{align*},
\begin{align*}y = a\brak{\sin \theta - \theta\cos \theta}\end{align*}at 
any point '$\theta$' is such that \hfill$\brak{1983 - 1 Mark}$

\begin{enumerate}
	\item it makes  constant angle with the x - axis
	\item it passes through the origin
	\item it is at a constant distance from the origin
	\item none of these\\
\end{enumerate}

	\item If \begin{align*}y=a\ln x + bx^2 +x\end{align*} has its extremum values at 
$x = -1$ and $x = 2$, then\hfill$\brak{1983 - 1 Mark}$

\begin{enumerate}
\begin{multicols}{2}
	\item $a = 2$, $b = -1$
	\item $a = 2$, b = $\frac{-1}{2}$
	\item $a = -2$, b = $\frac{1}{2}$
	\item none of these
\end{multicols}
\end{enumerate}

        \item Which one of the following curves cut the parabola
$y^2 = 4ax$ at right angles?\hfill$\brak{1994}$

\begin{enumerate}
\begin{multicols}{2}
	\item $x^2 + y^2 = a^2$
        \item $e^{\frac{-x}{2a}}$
	\item $y = ax$
	\item $x^2 = 4ay$
\end{multicols}
\end{enumerate}

	\item The function defined by 
\begin{align*}f\brak{x} = \brak{x+2}e^{-x}\end{align*} is

		\hfill$\brak{1994}$

\begin{enumerate}
	\item decreasing for all x
	\item decreasing in $\brak{-\infty, -1}$ and increasing
		in $\brak{-1, \infty}$
        \item increasing for all x
	\item decreasing in $\brak{-1, \infty}$ and increasing
		in $\brak{-\infty, -1}$\\
\end{enumerate}

	\item The function \begin{align*}f\brak{x} =
	\frac{\ln \brak{\pi + x}}{\ln \brak{e + x}}\end{align*} is

                               \hfill$\brak{1995S}$\\

\begin{enumerate}
	\item increasing on $\brak{0, \infty}$
	\item decreasing on $\brak{0, \infty}$
	\item increasing on $\brak{0, \frac{\pi}{e}}$,
		decreasing on \begin{align*}\brak{\frac{\pi}{e}, \infty}\end{align*}
	\item decreasing on $\brak{0, \frac{\pi}{e}}$,
		increasing on \begin{align*}\brak{\frac{\pi}{e}, \infty}\end{align*}\\
\end{enumerate}

	\item On the interval [0, 1] the function\begin{align*}x^{25}\brak{1-x}^{25}\end{align*} takes 
its maximum value at the point\hfill$\brak{1995S}$
\end{enumerate}

\begin{enumerate}
\begin{multicols}{4}
	\item 0 
	\item $\frac{1}{4}$ 
	\item $\frac{1}{2}$ 
        \item $\frac{1}{3}$
\end{multicols}
\end{enumerate}

\newpage
\bigskip
\renewcommand{\thefigure}
{\theenumi}
\renewcommand{\thetable}
{\theenumi}

\end{document}


