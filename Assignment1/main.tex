%iffalse
\let\negmedspace\undefined
\let\negthickspace\undefined
\documentclass[journal,12pt,onecolumn]{IEEEtran}
\usepackage{cite}
\usepackage{amsmath,amssymb,amsfonts,amsthm}
\usepackage{algorithmic}
\usepackage{graphicx}
\usepackage{textcomp}
\usepackage{xcolor}
\usepackage{txfonts}
\usepackage{listings}
\usepackage{enumitem}
\usepackage{mathtools}
\usepackage{gensymb}
\usepackage{comment}
\usepackage[breaklinks=true]{hyperref}
\usepackage{tkz-euclide} 
\usepackage{listings}  
\usepackage{gvv}
%\def\inputGnumericTable{}

\usepackage[latin1]{inputenc}                                
\usepackage{color}                                            
\usepackage{array}                                            
\usepackage{longtable}                                       
\usepackage{calc}                                             
\usepackage{multirow}
\usepackage{multicol}
\usepackage{hhline}                                           
\usepackage{ifthen}                                           
\usepackage{lscape}
\usepackage{tabularx}
\usepackage{array}
\usepackage{float}


\newtheorem{theorem}{Theorem}[section]
\newtheorem{problem}{Problem}
\newtheorem{proposition}{Proposition}[section]
\newtheorem{lemma}{Lemma}[section]
\newtheorem{corollary}[theorem]{Corollary}
\newtheorem{example}{Example}[section]
\newtheorem{definition}[problem]{Definition}
\newcommand{\BEQA}{\begin{eqnarray}}
\newcommand{\EEQA}{\end{eqnarray}}
\newcommand{\define}{\stackrel{\triangle}{=}}
\theoremstyle{remark}
\newtheorem{rem}{Remark}

% Marks the beginning of the document
\begin{document}
\bibliographystyle{IEEEtran}
\vspace{3cm}

\title{Assignment(matrix theory)}
\author{ee24btech11056 - S.Kavya Anvitha}
\maketitle
\section{A Fill in the blanks}

\begin{enumerate}
\item The larger of $\cos \brak{\ln \theta}$ and 
$\ln \brak{\cos \theta}$ if  $e^{\frac{-\pi}{2}}< \theta< \frac{\pi}{2}$
is \dots

\hfill$\brak{1983 - 1 Mark}$


\item The function $y=2x^{2}-\ln \abs x$
is monotonically increasing for values of $x\brak{\neq0}$ satisfying
the inequalities \dots and monotonically decreasing for values of $x$
satisfying the inequalities \dots

\hfill$\brak{1983 - 2 Marks}$
	
\item The set of all $x$ for which $\ln \brak{1+x} \leq x$
is equal to \dots
\hfill$\brak{1987 - 2 Marks}$

\item Let $\vec P$ be a variable point on the ellipse
$\frac{x^2}{a^2}+\frac{y^2}{b^2} = 1$
with foci $F_1$ and $F_2$. If $A$ is the area of the triangle $PF_1F_2$ 
then the maximum value of $A$ is \dots
\hfill$\brak{1994 - 2 Marks}$

\item Let $C$ be the curve $y^3 - 3xy + 2 = 0$ If $\vec H$ is the set 
of points on the curve $C$ where the tangent is horizontal and $\vec V$ is
the set of the point on the curve $C$ where the tangent is vertical
then $H =$ \dots and $V =$ \dots
\hfill$\brak{1994 - 2 Marks}$
\end{enumerate}

\section{B True / False}

\begin{enumerate}

\item If $x-r$ is a factor of the polynomial
$f\brak{x} = a_{n}x^{4}+\dots+a_{0}$,repeated $m$ times $\brak{1< m\leq n}$, 
then $r$ is a root of $f^\prime\brak{x}=0$ repeated $m$ times.

\hfill$\brak{1983 - 1 Mark}$

\item For $0 < a < x$, the minimum value of the function 
$log_a x + log_x a$ is $2$. 

\hfill$\brak{1984 - 1 Mark}$

\end{enumerate}

\section{C MCQs with One Correct Answer}

\begin{enumerate}

\item If $a+b+c = 0$, then the quadratic equation $3ax^2 + 2bx + c = 0$
has \hfill$\brak{1983 - 1 Mark}$
\begin{enumerate}
	\item at least one root in\sbrak{0,1}
	\item one root in \sbrak{2,3} and other in \sbrak{-2,-1}
        \item imaginary roots
	\item none of these
\end{enumerate}

 \item $AB$ is a diameter of a circle and $\vec C$ is any point on the
circumference of the circle. Then

\hfill$\brak{1983 - 1 Mark}$
\begin{enumerate}
	\item the area of$\Delta ABC$ is maximum when it is isosceles
	\item the area of $\Delta ABC$ is minimum when it is isosceles
	\item the perimeter of $\Delta ABC$ is minimumwhen it is isosceles
	\item none of these
\end{enumerate}

\item The normal to the curve 
\begin{align*}x = a\brak{\cos \theta + \theta\sin \theta}
\end{align*}
\begin{align*}y = a\brak{\sin \theta - \theta\cos \theta}
\end{align*}
at any point '$\theta$' is such that \hfill$\brak{1983 - 1 Mark}$
\begin{enumerate}
	\item it makes  constant angle with the $x$ - axis
	\item it passes through the origin
	\item it is at a constant distance from the origin
	\item none of these
\end{enumerate}

\item If $y=a\ln x + bx^2 +x$ has its extremum values at 
$\vec x = -1$ and $\vec x = 2$, then
\hfill$\brak{1983 - 1 Mark}$

\begin{enumerate}
\begin{multicols}{2}
	\item $ a = 2$, $b = -1$
	\item $a = 2$, b = $\frac{-1}{2}$
	\item $a = -2$, b = $\frac{1}{2}$
	\item none of these
\end{multicols}
\end{enumerate}

\item Which one of the following curves cut the parabola
$y^2 = 4ax$ at right angles?
\hfill$\brak{1994}$
\begin{enumerate}
\begin{multicols}{2}
	\item $x^2 + y^2 = a^2$
        \item $e^{\frac{-x}{2a}}$
	\item $y = ax$
	\item $x^2 = 4ay$
\end{multicols}
\end{enumerate}

\item The function defined by 
$f\brak{x} = \brak{x+2}e^{-x}$ is

\hfill$\brak{1994}$
\begin{enumerate}
	\item decreasing for all $x$
	\item decreasing in $\brak{-\infty, -1}$ and increasing
		in $\brak{(-1, \infty)}$
        \item increasing for all $x$
	\item decreasing in $\brak{(-1, \infty)}$ and increasing
		in $\brak{(-\infty, -1)}$
\end{enumerate}

\item The function 
\begin{align*}
		f\brak{x} =\frac{\ln \brak{\pi + x}}{\ln \brak{e + x}}
\end{align*} is
\hfill$\brak{1995S}$

\begin{enumerate}
	\item increasing on $\brak{0, \infty}$
	\item decreasing on $\brak{0, \infty}$
	\item increasing on $\brak{0, \frac{\pi}{e}}$,
		decreasing on $\brak{\frac{\pi}{e}, \infty}$
	\item decreasing on $\brak{0, \frac{\pi}{e}}$,
		increasing on $\brak{\frac{\pi}{e}, \infty}$
\end{enumerate}

\item On the interval \sbrak{0, 1} the function $x^{25}\brak{1-x}^{25}$
takes its maximum value at the point 
\hfill$\brak{1995S}$
\end{enumerate}

\begin{enumerate}
\begin{multicols}{4}
	\item $0$ 
	\item $\frac{1}{4}$ 
	\item $\frac{1}{2}$ 
        \item $\frac{1}{3}$
\end{multicols}
\end{enumerate}

\newpage
\bigskip
\renewcommand{\thefigure}
{\theenumi}
\renewcommand{\thetable}
{\theenumi}

\end{document}


