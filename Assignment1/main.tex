%iffalse
\let\negmedspace\undefined
\let\negthickspace\undefined
\documentclass[journal,12pt,twocolumn]{IEEEtran}
\usepackage{cite}
\usepackage{amsmath,amssymb,amsfonts,amsthm}
\usepackage{algorithmic}
\usepackage{graphicx}
\usepackage{textcomp}
\usepackage{xcolor}
\usepackage{txfonts}
\usepackage{listings}
\usepackage{enumitem}
\usepackage{mathtools}
\usepackage{gensymb}
\usepackage{comment}
\usepackage[breaklinks=true]{hyperref}
\usepackage{tkz-euclide} 
\usepackage{listings}  
\usepackage{gvv}
%\def\inputGnumericTable{}

\usepackage[latin1]{inputenc}                                
\usepackage{color}                                            
\usepackage{array}                                            
\usepackage{longtable}                                       
\usepackage{calc}                                             
\usepackage{multirow}                                         
\usepackage{hhline}                                           
\usepackage{ifthen}                                           
\usepackage{lscape}
\usepackage{tabularx}
\usepackage{array}
\usepackage{float}


\newtheorem{theorem}{Theorem}[section]
\newtheorem{problem}{Problem}
\newtheorem{proposition}{Proposition}[section]
\newtheorem{lemma}{Lemma}[section]
\newtheorem{corollary}[theorem]{Corollary}
\newtheorem{example}{Example}[section]
\newtheorem{definition}[problem]{Definition}
\newcommand{\BEQA}{\begin{eqnarray}}
\newcommand{\EEQA}{\end{eqnarray}}
\newcommand{\define}{\stackrel{\triangle}{=}}
\theoremstyle{remark}
\newtheorem{rem}{Remark}

% Marks the beginning of the document
\begin{document}
\bibliographystyle{IEEEtran}
\vspace{3cm}

\title{Assignment(matrix theory)}
\author{ee24btech11056 - S.Kavya Anvitha}
\maketitle
\textbf{A Fill in the blanks}\\[6pt]
1.  The larger of $\cos (\ln \theta)$ and  $\ln (\cos \theta)$ if $e^{\frac{-\pi}{2}}< \theta< \frac{\pi}{2}$\\[2pt] 
\indent is ..........\hspace{4.3cm}(1983 - 1 Mark)\\[3pt]
 2.  The function $y=2x^{2}-\ln |x|$ is monotonically \indent increasing for values of $x(\neq0)$ satisfying the \indent inequalities ...... and monotonically decreasing for \indent values of x satisfying the inequalities ............\\[2pt]\indent\hspace{5.5cm}(1983 - 2 Marks)\\[3pt]
3.  The set of all x for which $\ln (1+x) \leq x$ is equal \indent to .........\hspace{4cm}(1987 - 2 Marks)\\[3pt]
4.  Let P be a variable point on the ellipse $\frac{x^2}{a^2}+\frac{y^2}{b^2} = 1$ \indent with foci $F_1$ and $F_2$. If A is the area of the \indent triangle P$F_1$$F_2$ then the maximum value of A is \hspace{5cm}\indent............. \hspace{3.6cm}(1994 - 2 Marks)\\[3pt]
5.  Let C be the curve $y^3$ - 3xy + 2 = 0. If H is the \indent set of points on the curve C where the tangent \indent is horizontal and V is the set of the point on the \indent curve C where the tangent is vertical then H = \indent ......... and V = .........\hspace{1.7cm}(1994 - 2 Marks)\\[6pt]
\indent\hspace{0.3cm}\textbf{B True / False}\\[6pt]
1. If x-r is a factor of the polynomial f(x) = \indent $a_n$$x^4$+....+$a_0$, repeated m times $(1< m\leq n)$, then \indent r is a root of f'(x)=0 repeated m times.\\[2pt]
\indent \hspace{5.5cm}(1983 - 1 Mark)\\[3pt]
2.  For $0 < a < x$, the minimum value of the function \indent $log_a x + log_x a$ is 2. \hspace{2.3cm}(1984 - 1 Mark)\\[6pt]
\indent\hspace{0.3cm}\textbf{C MCQs with One Correct Answer}\\[6pt]
1.  If a+b+c = 0, then the quadratic equation 3a$x^2$ \indent + 2bx + c = 0 has\hspace{2.2cm}(1983 - 1 Mark)\\[2pt]\indent (a)\hspace{0.2cm} at least one root in [0,1]\\[2pt]\indent (b)\hspace{0.2cm} one root in [2,3] and other in [-2,-1]\\[2pt]\indent (c)\hspace{0.2cm} imaginary roots\\[2pt]\indent (d)\hspace{0.2cm} none of these\\[3pt]
2.  AB is a diameter of a circle and C is any point \indent on the circumference of the circle. Then\\[2pt]\indent\hspace{5.4cm} (1983 - 1 Mark)\\[2pt]\indent (a)\hspace{0.2cm} the area of $\Delta$ ABC is maximum when it is \indent \indent \indent isosceles\\[2pt]\indent (b)\hspace{0.2cm} the area of $\Delta$ ABC is minimum when it is \indent \indent \indent isosceles\\[2pt]\indent (c)\hspace{0.2cm} the perimeter of $\Delta$ ABC is minimum when \indent \indent \indent it is isosceles\\[2pt]\indent (d)\hspace{0.2cm} none of these\\[3pt]
3.  The normal to the curve x = a($\cos \theta$ + $\theta\sin \theta$), \indent $y = a(\sin \theta$ - $\theta\cos \theta$) at any point '$\theta$' is such that\\[2pt]\indent\hspace{5.4cm}(1983 - 1 Mark)\\[2pt]\indent (a)\hspace{0.2cm} it makes  constant angle with the x - axis\\[2pt]\indent (b)\hspace{0,2cm} it passes through the origin\\[2pt]\indent (c)\hspace{0.2cm} it is at a constant distance from the origin\\[2pt]\indent (d)\hspace{0.2cm} none of these\\[3pt]
4.  If $y=a\ln x + bx^2 +x$ has its extremum values at \indent x = -1 and x = 2, then\hspace{1.6cm}(1983 - 1 Mark)\\[2pt]\indent (a)\hspace{0.2cm} a = 2, b = -1\hspace{1cm} (b)\hspace{0.2cm} a = 2, b = $\displaystyle\frac{-1}{2}$\\[2pt]\indent (c)\hspace{0.2cm} a = -2, b = $\displaystyle\frac{1}{2}$\hspace{1cm} (d)\hspace{0.2cm} none of these\\[3pt]
5. Which one of the following curves cut the \indent  parabola $y^2 = 4ax$ at right angles?\hspace{1.2cm}(1994)\\[2pt]\indent (a)\hspace{0.2cm} $x^2 + y^2 = a^2$\hspace{2cm} (b)\hspace{0.2cm} $e^{\frac{-x}{2a}}$\\[2pt]\indent (c)\hspace{0.2cm} y = ax\hspace{2.9cm} (d)\hspace{0.2cm} $x^2 = 4ay$\\[3pt]
6.  The function defined by f(x) = (x+2)$e^{-x}$ is\hspace{0.1cm}(1994)\\[2pt]\indent (a)\hspace{0.2cm} decreasing for all x\\[2pt]\indent (b)\hspace{0.2cm} decreasing in $(-\infty, -1)$ and increasing in \indent \indent \indent $(-1, \infty)$\\[2pt]\indent (c)\hspace{0.2cm} increasing for all x\\[2pt]\indent (d)\hspace{0.2cm} decreasing in $(-1, \infty,)$ and increasing in \indent \indent \indent $(-\infty, -1)$\\[3pt]
7.  The function f(x) = $\displaystyle\frac{\ln (\pi + x)}{\ln (e + x)}$ is\hspace{1.5cm}(1995S)\\[2pt]\indent (a)\hspace{0.2cm} increasing on $(0, \infty)$\\[2pt]\indent (b)\hspace{0.2cm} decreasing on $(0, \infty)$\\[2pt]\indent (c)\hspace{0.2cm} increasing on $(0, \displaystyle\frac{\pi}{e})$, decreasing on $(\displaystyle\frac{\pi}{e}, \infty)$\\[2pt]\indent (d)\hspace{0.2cm} decreasing on $(0, \displaystyle\frac{\pi}{e})$, increasing on $(\displaystyle\frac{\pi}{e}, \infty)$\\[3pt]
8.  On the interval [0, 1] the function $x^{25}(1-x)^{25}$ \indent takes its maximum value at the point\hspace{0.7cm}(1995S)\\[2pt]\indent (a)\hspace{0.2cm} 0\hspace{0.9cm} (b)\hspace{0.2cm} $\displaystyle\frac{1}{4}$\hspace{0.9cm} (c)\hspace{0.2cm} $\displaystyle\frac{1}{2}$\hspace{0.9cm} (d)\hspace{0.2cm} $\displaystyle\frac{1}{3}$
    
\newpage
\bigskip
\renewcommand{\thefigure}
{\theenumi}
\renewcommand{\thetable}
{\theenumi}

\end{document}
