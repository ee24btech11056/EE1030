\let\negmedspace\undefined
\let\negthickspace\undefined
\documentclass[journal]{IEEEtran}
\usepackage[a5paper, margin=10mm, onecolumn]{geometry}
%\usepackage{lmodern} % Ensure lmodern is loaded for pdflatex
\usepackage{tfrupee} % Include tfrupee package

\setlength{\headheight}{1cm} % Set the height of the header box
\setlength{\headsep}{0mm}     % Set the distance between the header box and the top of the text
\usepackage{multicol}
\usepackage{gvv-book}
\usepackage{gvv}
\usepackage{cite}
\usepackage{amsmath,amssymb,amsfonts,amsthm}
\usepackage{algorithmic}
\usepackage{graphicx}
\usepackage{textcomp}
\usepackage{xcolor}
\usepackage{txfonts}
\usepackage{listings}
\usepackage{enumitem}
\usepackage{mathtools}
\usepackage{gensymb}
\usepackage{comment}
\usepackage[breaklinks=true]{hyperref}
\usepackage{tkz-euclide} 
\usepackage{pgfplots}
\pgfplotsset{compat=1.18}
\usepackage{listings}
% \usepackage{gvv}  
\newcommand{\questionref}[1]{{ #1}}
\usepackage{xparse}                                      
\def\inputGnumericTable{}                                 
\usepackage[latin1]{inputenc}                                
\usepackage{color}                                            
\usepackage{array}                                            
\usepackage{longtable}                                       
\usepackage{calc}                                             
\usepackage{multirow}                                         
\usepackage{hhline}                                           
\usepackage{ifthen}                                           
\usepackage{lscape}
\usepackage{tikz}
% Marks the beginning of the document
\begin{document}
\bibliographystyle{IEEEtran}
\vspace{3cm}

\title{2013-AE}
\author{EE24BTECH11056 - S.Kavya Anvitha}
\maketitle
%\newpage
\bigskip

\renewcommand{\thefigure}{\theenumi}
\renewcommand{\thetable}{\theenumi}
\begin{enumerate}

\item A second identical airfoil is placed behind the first one at a distance of $c/2$ from the trailing edge of the first. The second airfoil has an unknown circulation $\Gamma_2$, placed at its quarter chord. The normal velocity becomes zero at the same chord-wise locations of the respective airfoils as in the previous question. The values of $\Gamma_1$ and $\Gamma_2$ are respectively

\begin{enumerate}
    \item $\frac{4}{3} \pi c U \alpha, \; \frac{2}{3} \pi c U \alpha$
    \item $\frac{2}{3} \pi c U \alpha, \; \frac{2}{3} \pi c U \alpha$
    \item $\frac{2}{3} \pi c U \alpha, \; \frac{1}{3} \pi c U \alpha$
    \item $\frac{4}{3} \pi c U \alpha, \; \frac{4}{3} \pi c U \alpha$\\
\end{enumerate}
\section{Statement for Linked Answer Questions \questionref{54} and \questionref{55}:} A wing-body alone configuration airplane with a wing loading of $\frac{W}{S} = 1000N/m^2$ is flying in cruise condition at a speed $V = 90m/s$ at sea-level (air density at sea-level $\rho_e = 1.122kg/m^3$). The zero lift pitching moment coefficient of the airplane is $C_{\text{m0}} = C_{m_{\text{ac}}} = -0.06$ and the location of airplane aerodynamic center from the wing leading edge is $X_{\text{ac}} = 0.25c$. Here $c$ is the chord length.
\item The airplane trim lift coefficient $C_{L_{\text{trim}}}$ is

\begin{enumerate}
    \item $0.502$
    \item $0.402$
    \item $0.302$
    \item $0.202$\\
\end{enumerate}
\item Distance of center of gravity of the aircraft $X_{\text{CG}}$ from the wing leading edge is

\begin{enumerate}
    \item $0.447c$
    \item $-0.547c$
    \item $0.547c$
    \item $-0.25c$\\
\end{enumerate}

\textbf{General Aptitude (GA) Questions}

Q.56 -- Q.60 carry one mark each.\\

\item If $3 \leq X \leq 5$ and $8 \leq Y \leq 11$ then which of the following options is TRUE?

\begin{enumerate}
    \item $\frac{3}{5} \leq \frac{X}{Y} \leq \frac{8}{5}$
    \item $\frac{3}{11} \leq \frac{X}{Y} \leq \frac{5}{8}$
    \item $\frac{3}{5} \leq \frac{X}{Y} \leq \frac{8}{5}$
    \item $\frac{3}{5} \leq \frac{X}{Y} \leq \frac{8}{11}$\\
\end{enumerate}

\item The Headmaster \underline to speak to you.

Which of the following options is incorrect to complete the above sentence?

\begin{enumerate}
    \item is wanting
    \item wants
    \item want
    \item was wantng\\
\end{enumerate}

\item Mahatma Gandhi was known for his humility as

\begin{enumerate}
    \item he played an important role in humiliating exit of British from India.
    \item he worked for humanitarian causes.
    \item he displayed modesty in his interactions.
    \item he was a fine human being.\\
\end{enumerate}
\item \underline{All engineering students} \underline{should learn mechanics}, \underline{mathematics and} \underline{how to do computation}.

Which of the above underlined parts of the sentence is not appropriate?

\begin{enumerate}
    \item I
    \item II
    \item III
    \item IV\\
\end{enumerate}

\item Select the pair that best expresses a relationship similar to that expressed in the pair: \textbf{water: pipe}

\begin{enumerate}
    \item cart: road
    \item electricity: wire
    \item sea: beach
    \item music: instrument\\
\end{enumerate}

Q. 61 to Q. 65 carry two marks each.\\

\item Velocity of an object fired directly in upward direction is given by $V = 80 - 32t$, where $t$ (time) is in seconds. When will the velocity be between $32 m/sec$ and $64 m/sec$?

\begin{enumerate}
    \item (1, 3/2)
    \item (1/2, 1)
    \item (1/2, 3/2)
    \item (1, 3)\\
\end{enumerate}

\item In a factory, two machines $M1$ and $M2$ manufacture $60\%$ and $40\%$ of the autocomponents respectively. Out of the total production, $2\%$ of $M1$ and $3\%$ of $M2$ are found to be defective. If a randomly drawn autocomponent from the combined lot is found defective, what is the probability that it was manufactured by $M2$?

\begin{enumerate}
    \item $0.35$
    \item $0.45$
    \item $0.5$
    \item $0.4$\\
\end{enumerate}

\item Following table gives data on tourists from different countries visiting India in the year 2011.

\begin{table}[h!]    
  \centering
  \begin{tabular}[12pt]{ |c| c|}
    \hline
    \textbf{Variable} & \textbf{Description}\\ 
    \hline
    $A,B$ & End points of diameter \\
    \hline
    $O$ & centre of the circle \\
    \hline
\end{tabular}

\end{table}

Which two countries contributed to the one third of the total number of tourists who visited India in 2011?

\begin{enumerate}
    \item USA and Japan
    \item USA and Australia
    \item England and France
    \item Japan and Australia\\
\end{enumerate}

\item If $| -2x + 9 | = 3$ then the possible value of $| -x| - x^2$ would be:

\begin{enumerate}
    \item $30$
    \item $-30$
    \item $-42$
    \item $42$\\
\end{enumerate}
\item All professors are researchers\\
Some scientists are professors\\

Which of the given conclusions is logically valid and is inferred from the above arguments?

\begin{enumerate}
    \item All scientists are researchers
    \item All professors are scientists
    \item Some researchers are scientists
    \item No conclusion follows
\end{enumerate}


\end{enumerate}
\end{document}i
